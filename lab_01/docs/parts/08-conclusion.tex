\chapter*{Заключение}
\addcontentsline{toc}{chapter}{Заключение}
В данной лабораторной работе были рассмотрено расстояние Левенштейна. Данное расстояние показывает минимальное количество операций (вставка, удаление, замены), которое необходимо для перевода одной строки в другую. Это расстояние помогает определить схожесть двух строк.

Впервые задачу поставил в 1965 году советский математик Владимир Левенштейн при изучении последовательностей 0-1, впоследствии более общую задачу для произвольного алфавита связали с его именем.

Расстояние Левенштейна применяется в теории информации и компьютерной лингвистике для:
\begin{itemize}
	\item исправления ошибок в слове(в поисковых системах, базах данных, при вводе текста, при автоматическом распознавании отсканированного текста или речи);
	\item сравнения текстовых файлов утилитой diff;
	\item для сравнения генов, хромосом и белков в биоинформатике.
\end{itemize}

Цели данной лабораторной работы были достигнуты, а именно изучение и исследование особенностей задач динамического программирования на алгоритмах Левенштейна и Дамерау-Левенштейна.

Для поставленной целей были выполнены следующие задачи:
\begin{enumerate}[label={\arabic*)}]
	\item изучено расстояния Левенштейна и Дамерау-Левенштейна;
	\item создано ПО, реализующее следующие алгоритмы:
	\begin{itemize}
		\item нерекурсивный метод поиска расстояния Левенштейна;
		\item нерекурсивный метод поиска Дамерау-Левенштейна;
		\item рекурсивный метод поиска Дамерау-Левенштейна;
		\item  рекурсивный с кешированием метод поиска Дамерау-Левенштейна.
	\end{itemize}
	\item выбраны инструменты для замера процессорного времени выполнения реализаций алгоритмов;
	\item Проведены анализ затрат работы программы по времени и по памяти, выяснить влияющие на них характеристики. 
\end{enumerate}