\addcontentsline{toc}{chapter}{Список используемой литературы}
\renewcommand\bibname{Список используемой литературы} % переименовать страницу списка литературы

\begin{thebibliography}{5}
	\bibitem{book_lev}
	Левенштейн В. И. Двоичные коды с исправлением выпадений, вставок и замещений символов. – М.: Доклады АН СССР, 1965: Т. 163. С. 845– 848.
	\bibitem{cpp}
	Документация по Microsoft C++ [Электронный ресурс]. 
	
	Режим доступа: \url{https://learn.microsoft.com/ru-ru/cpp/?view=msvc-170&viewFallbackFrom=vs-2017} (дата обращения: 25.09.2022).
	\bibitem{time}
	C library function clock() [Электронный ресурс]. 
	
	Режим доступа: \url{https://www.tutorialspoint.com/c_standard_library/c_function_clock.htm} (дата обращения: 25.09.2022).
	\bibitem{windows}
	Windows 10 Pro 2h21 64-bit  Электронный ресурс]. 
	
	Режим доступа: \url{https://www.microsoft.com/ru-ru/software-download/windows10} (дата обращения: 25.09.2022).
	\bibitem{intel}
	Intel [Электронный ресурс]. 
	
	Режим доступа: \url{https://ark.intel.com/content/www/ru/ru/ark/products/201839/intel-core-i510300h-processor-8m-cache-up-to-4-50-ghz.html} (дата обращения: 25.09.2022).
\end{thebibliography}