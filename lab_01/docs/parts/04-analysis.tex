\chapter{Аналитическая часть}
\section{Расстояние Левенштейна}

Расстояние Левенштейна \cite{levenshtein} (редакционное расстояние, дистанция редактирования) --- метрика, измеряющая разность между двумя последовательностями символов.

\noindent Стоимость операций могут зависеть от вида операций:
\begin{enumerate}
	\item $w(a, b)$ --- цена замены символа $a$ на $b$, R (от англ. replace);
	\item $w(\lambda, b)$ --- цена вставки символа $b$, I (от англ. insert);
	\item $w(a, \lambda)$ --- цена удаления символа $a$, D (от англ. delete).
\end{enumerate}

Будем считать стоимость каждой вышеизложенной операции равной 1, то есть:
\begin{itemize}
	\item $w(a, b) = 1$, $a \neq b$;
	\item $w(\lambda, b) = 1$;
	\item $w(a, \lambda) = 1$.
\end{itemize}

Введем понятие совпадения символов --- M (от англ. match). Его стоимость будет равна 0, то есть $w(a, a) = 0$.

Расстояние Левенштейна между двумя строками $S_{1}$ и $S_{2}$, длинной M и N соответственно. Расстояние Левенштейна рассчитывается по рекуррентной формуле \ref{eq:L}:

\begin{equation}
	\label{eq:L}
	D(i, j) =
	\begin{cases}
		0, &\text{i = 0, j = 0}\\
		i, &\text{j = 0, i > 0}\\
		j, &\text{i = 0, j > 0}\\
		min \begin{cases}
			D(i, j - 1) + 1\\
			D(i - 1, j) + 1\\
			D(i - 1, j - 1) +  m(S_{1}[i], S_{2}[j]) \\
		\end{cases}
		&\text{i > 0, j > 0}
	\end{cases}
\end{equation}

Сравнение символов строк $S_1$ и $S_2$ рассматривается в формуле \ref{eq:m}:

\begin{equation}
	\label{eq:m}
	m(a, b) = \begin{cases}
		0 &\text{если a = b,}\\
		1 &\text{иначе}
	\end{cases}.
\end{equation}

\subsection{Рекурсивный алгоритм нахождения расстояния Левенштейна}

Рекурсивный алгоритм реализует формулу \ref{eq:L}, функция $D$ составлена таким образом, что:
\begin{enumerate}
	\item  Для передачи из пустой строки в пустую требуется ноль операций;
	\item Для перевода из пустой строки в строку $a$ требуется $|a|$ операций;
	\item Для перевода из строки $a$ в пустую строку требуется $|a|$ операций;
	\item Для перевода из строки $a$ в строку $b$ требуется выполнить последовательно некоторое количество операций(удаление, вставка, замена) в некоторой последовательности. Последовательность поведения любых двух операций можно поменять, порядок поведения операций не имеет никакого значения. Полагая, что $a'$, $b'$ - строки $a$ и $b$ без последнего символа соответственно, цена преобразования из строки $a$ в строку $b$ может быть выражена как:
	\begin{itemize}
		\item стоимость преобразования строки $a$ в $b$ и стоимость проведения операции удаления, которая необходима для преобразования $a'$ в $a$;
		\item стоимость преобразования строки $a$ в $b$ и стоимость проведения операции вставки, которая необходима для преобразования $b'$ в $b$;
		\item стоимость преобразования из $a'$ в $b'$ и операции замены, предполагая, что $a'$ и $b'$ оканчиваются разные символы;
		\item стоимость преобразования из $a'$ в $b'$ , предполагая, что $a$ и $b$ оканчиваются на один и тот же символ.
	\end{itemize}
\end{enumerate}

Минимальной стоимостью преобразования будет минимальное значение приведенных вариантов.

\subsection{Рекурсивный алгоритм нахождения расстояния Левенштейна с кешированием}

Рекурсивная реализация алгоритма Левенштейна малоэффективна по времени при больших $i, j$, по причине проблемы повторных вычислений $D(i,j)$. Для оптимизации нахождения расстояния Левенштейна можно использовать матрицу в целях хранения соответствующих промежуточных значений. В таком случае алгоритм представляет собой рекурсивное заполнение матрицы $A_{|a|,|b|}$ значениями $D(i,j)$, такое хранение промежуточных данных можно назвать кэшем для рекурсивного алгоритма.

\subsection{Нерекурсивный алгоритм нахождения расстояния Левенштейна}

Рекурсивная реализация алгоритма Левенштейна с кешированием малоэффективна по времени при больших $i, j$. Для оптимизации можно использовать итерационную реализацию заполнение матрицы промежуточными значениями $D(i,j)$.

Однако матричный алгоритм является малоэффективным по памяти при больших $i, j$, т.к. множество промежуточных значений $D(i,j)$ хранится в памяти после их использования. Для оптимизации нахождения расстояния Левенштейна можно использовать кэш, т.е. пару строк, содержащую значения $D(i,j)$, вычисленные в предыдущей итерации алгоритма и значения $D(i,j)$, вычисляемый в текущей итерации.

\section{Расстояние Дамерау-Левенштейна}
Расстояние Дамерау-Левенштейна(названо в честь ученых Фредерика Дамерау и Владимир Левенштейна) - это мера разницы двух строк символов, определяемая как минимальное количество операций вставки, удаления, замены и транспозиций (перестановки двух соседних символов), необходимых для перевода одной строки в другую. Является модификацией расстояния Левенштейна, то есть к операциям добавляется операция транспозиция $T$ (от англ. transposition).

Расстояние Дамерау-Левенштейна может быть вычислено по рекуррентной формуле:

\begin{equation}
	\label{eq:DL}
	D(i, j) = 
	\begin{cases}
		0, &\text{i = 0, j = 0}\\
		i, &\text{j = 0, i > 0}\\
		j, &\text{i = 0, j > 0}\\
		min \begin{cases}
			D(i, j - 1) + 1\\
			D(i - 1, j) + 1\\
			D(i - 1, j - 1) + m(S_{1}[i], S_{2}[j]) \\
			D(i - 2, j - 2) + m(S_{1}[i], S_{2}[j]) \\
		\end{cases}
		& \begin{aligned}
			& \text{если i > 1, j > 1} \\
			& S_{1}[i] = S_{2}[j - 1] \\
			& S_{1}[i - 1] = S_{2}[j] \\
		\end{aligned}\\
		min \begin{cases}
			D(i, j - 1) + 1\\
			D(i - 1, j) + 1 \\
			D(i - 1, j - 1) + m(S_{1}[i], S_{2}[j]) \\
		\end{cases}
		 & \text{иначе}
	\end{cases}
\end{equation}


\section*{Вывод}

В данном разделе были рассмотрены алгоритмы динамического программирования - алгоритмы нахождения расстояния Левенштейна и Дамерау-Левенштейна, формулы которых задаются рекуррентно, а следовательно, данные алгоритмы могут быть реализованы рекурсивно и итеративно. На вход алгоритмам поступают две строки, которые могут содержать как русские, так и английские буквы, также будет предусмотрен ввод пустых строк.
