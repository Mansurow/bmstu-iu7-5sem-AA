\chapter{Технологическая часть}

В данном разделе рассмотрены средства реализации, а также представлены листинги реализаций алгоритма расчета термовой частота для всех термов из выборки документов.

\section{Средства реализации}

В данной работе для реализации был выбран язык программирования $Python$ \cite{pythonlang}. 
Данный выбор обусловлен соответствием с требованиями выдвинутыми в конструкторской части, а именно в языке программирование $Python$ имеется встроенные типы данных --- словарь и массив и инструменты для поиска подстроки в строке.

\section{Сведения о модулях программы}

Данная программа разбита на следующие модули:

Программа состоит из четырех модулей:
\begin{enumerate}[label=\arabic*)]
	\item $main.py$ --- файл, содержащий точку входа;
	\item $utils.py$ --- файл, содержащий служебные алгоритмы;
	\item $terms.py$ --- файл, содержавший термы и константы программы;
	\item $dish.py$ --- файл, содержащий код класса \textit{Dish}, формализуемого рассматриваемый объект. 
\end{enumerate}

\section{Реализация алгоритмов}

В листингах \ref{lst:check_req} -- \ref{lst:full_comb} приведены реализации алгоритма выделения наиболее информативных терминов для каждого документа. 
В качестве термов в данной реализации рассматриваются слова, состоящие из латинских букв. В качестве документов рассматриваются строки, состоящие из таких слов, пробелов и знаков пунктуации.

\lstinputlisting[label=lst:check_req,caption=Реализация проверки запроса на корректность, firstline=31,lastline=52]{../utils.py}

\clearpage

\lstinputlisting[label=lst:get_term,caption=Реализация поиска терма, firstline=55,lastline=82]{../utils.py}

\clearpage

\lstinputlisting[label=lst:get_data,caption=Реализация поиска интервала значений для объекта по терму, firstline=86,lastline=109]{../utils.py}

\clearpage

\lstinputlisting[label=lst:full_comb,caption=Реализация алгоритма поиска полным перебором, firstline=111,lastline=116]{../utils.py}

\section*{Вывод}
В данном разделе был представлен листинг рассматриваемого алгоритма поиска в словаре, приведена информация о средствах реализации, сведения о модулях программы.
