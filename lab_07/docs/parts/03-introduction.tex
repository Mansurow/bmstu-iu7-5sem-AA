\chapter*{Введение}
\addcontentsline{toc}{chapter}{Введение}

В процессе развития компьютерных систем количество обрабатываемых данных увеличивалось, вследствие чего множество операций над наборами данных стали выполняться очень долго, поскольку чаще всего это был обычный перебор. 
Это вызвало необходимость создать новые алгоритмы, которые решают поставленную задачу на порядок быстрее стандартного решения прямого обхода. 
В том числе это касается и словарей, в которых одной из основных операций является операция поиска.


Целью данной лабораторной работы является получение навыков поиска по словарю при ограничении на значение признака, заданном при помощи лингвистической переменной..

Для поставленной цели необходимо выполнить следующие задачи:
\begin{enumerate}
	\item формализовать объект по варианту и его признак;
	\item составить анкета для заполнения респондентом;
	\item провести анкетирование респондентов;
	\item построить функцию принадлежности термам числовых значений признака, описываемого лингвистической переменной, на основе статистической обработки мнений респондентов, выступающих в роли экспертов;
	\item описать алгоритм поиска в словаре объектов;
	\item описать структуру данных словаря;
	\item реализовать описанный алгоритм поиска в словаре;
	\item описать и обосновать результаты в виде отчета о выполненной лабораторной работе, выполненном как расчётно-пояснительная записка к работе.
\end{enumerate}