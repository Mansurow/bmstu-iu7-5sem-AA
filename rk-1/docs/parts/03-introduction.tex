\chapter*{Введение}
\addcontentsline{toc}{chapter}{Введение}

Во многих областях человеческой деятельности информацию часто представляют в форме матриц.
Матрица --- это регулярный числовой массив. 
В специальной литературе имеется несколько определений разреженной матрицы. 
Суть их состоит в том, что матрица разрежена, если в ней больше нулевых элементов, чем ненулевых.
Итак, сформулировались три основных идеи, которые направляли развитие большей части технологии разреженных
матриц~\cite{csr}: 
\begin{itemize}
	\item хранить только ненулевые элементы;
	\item оперировать только с ненулевыми элементами;
	\item сохранять разреженность.
\end{itemize}

Алгоритм, хранящий и обрабатывающий меньшее число нулей, более сложен, труднее программируется и целесообразен только для достаточно больших матриц.

Целью данной рубежного контроля является описание пошагового поиска строчной координаты элемента матрицы.

Для поставленной цели необходимо выполнить следующие задачи.
\begin{enumerate}
	\item Описать понятие разреженной матрицы.
	\item Описать алгоритм поиска строчной координаты элемента матрицы.
	\item Реализовать программу, реализующую пошаговую работу алгоритма с выводом информации.
\end{enumerate}