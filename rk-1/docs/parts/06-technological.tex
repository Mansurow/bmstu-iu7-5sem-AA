\chapter{Технологическая часть}

В данном разделе рассмотрены средства реализации, а также представлены листинги реализаций алгоритма расчета термовой частота для всех термов из выборки документов.

\section{Средства реализации}

В качестве языка программирования для реализации данной лабораторной работы был выбран язык $++C$~\cite{cpp-lang}. 
Данный выбор обусловлен тем  что реализация структур данных и алгоритмов разреженной матрицы лабораторной работы №5 было выполнено на данном языке.

\section{Реализация алгоритмов}

На листинге \ref{lst:csr_st} представлена структура данных $matrix_csr_t$, которая задает матрицу РСФ.
На листинге \ref{lst:find} представлен алгоритм поиска строчной координаты элемента матрицы с пошаговым выводом информации.

\lstinputlisting[label=lst:csr_st,caption=Реализация структуры данных матрицы РСФ, firstline=7,lastline=15]{../main.cpp}

\clearpage

\lstinputlisting[label=lst:find,caption=Реализация алгоритма поиска строчной коордтината элемента матрицы, firstline=72,lastline=103]{../main.cpp}

\clearpage


\section*{Вывод}

В данном разделе была приведена информация о выбранных средствах для разработки алгоритма и реализация алгоритма пошагового поиска строчной координаты . 