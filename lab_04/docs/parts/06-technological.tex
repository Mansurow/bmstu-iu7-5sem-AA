\chapter{Технологическая часть}

В данном разделе рассмотрены средства реализации, а также представлены листинги реализаций алгоритма расчета термовой частота для всех термов из выборки документов.

\section{Средства реализации}

В качестве языка программирования для реализации данной лабораторной работы был выбран язык $C$~\cite{c-lang}. Данный выбор обусловлен наличием у языка встроенной модулем измерения процессорного времени и соответствием с выдвинутыми  требованиям.

Для работы с потоками использовались функции модуля <pthread.h>~\cite{c-thread}. Для работы с сущностью вспомогательного потока необходимо воспользоваться функцией $pthread\_create()$ для создания потока и указания функции, которую начнет выполнять созданный поток. Далее при помощи вызова $pthread\_join()$ необходимо (в рамках данной лабораторной работы) дождаться завершения всех вспомогательных потоков, чтобы в главном потоке обработать результаты их работы.

\clearpage

\section{Сведения о модулях программы}

Данная программа разбита на следующие модули:

\begin{itemize}[label=---]
	\item \texttt{main.c} --- файл, содержащий точку входа в программу. В нем происходит общение с пользователем и вызов алгоритмов;
	\item \texttt{tf\_alg.c} --- файл содержит функции динамического выделения, очищения памяти и однопоточную реализацию алгоритма;
	\item \texttt{pthread\_tf.c} --- файл содержит функции многопоточной реализации алгоритма;
	\item \texttt{time\_meassure.c} --- файл содержит функции, замеряющее процессорное время методов однопоточной и многопоточной реализаций алгоритма.
\end{itemize}

\section{Реализация алгоритмов}

В листингах \ref{lst:tf_alg} -- \ref{lst:multy_tf_alg} приведены реализации алгоритма выделения наиболее информативных терминов для каждого документа. 
В качестве термов в данной реализации рассматриваются слова, состоящие из латинских букв. В качестве документов рассматриваются строки, состоящие из таких слов, пробелов и знаков пунктуации.

\clearpage

\lstinputlisting[label=lst:tf_alg,caption=Функция работы однопоточной реализации алгоритма расчета термовой частоты для каждого терма из выборки документов, firstline=85,lastline=121]{../src/tf_alg.c}

\clearpage

\lstinputlisting[label=lst:thread,caption=Функция работы одного вспомогательного потока, firstline=3,lastline=29]{../src/pthread_tf.c}

\clearpage

\lstinputlisting[label=lst:multy_tf_alg,caption={Функция работы основного потока, запускающего вспомогательные потоки}, firstline=33,lastline=59]{../src/pthread_tf.c}


\section*{Вывод}

В данном разделе были рассмотрены средства реализации, а также представлен листинг реализации алгоритма расчета термовой частоты для каждого терма из выборки документов.