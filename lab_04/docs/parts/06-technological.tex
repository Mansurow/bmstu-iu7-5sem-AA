\chapter{Технологическая часть}

В данном разделе рассмотрены средства реализации, а также представлены листинги реализации алгоритма выделения наиболее информативных терминов, по метрике TF, из выборки документов.

\section{Требования к программному обеспечению}

К программе предъявлены ряд требований:

\begin{itemize}[label=---]
	\item иметь интерфейс для выбора действий;
	\item динамически выделять память под массив данных;
	\item работа с массивами и <<нативными>> потоками;
	\item замерять процессорное время алгоритмов сортировки.
\end{itemize}

\section{Средства реализации}

В качестве языка программирования для реализации данной лабораторной работы был выбран язык $C$~\cite{c-lang}. Данный выбор обусловлен наличием у языка встроенной модулем измерения процессорного времени и соответствием с выдвинутыми техническими требованиям.

Для работы с потоками использовались функции модуля <pthread.h>~\cite{c-thread}. Для работы с сущностью вспомогательного потока необходимо воспользоваться функцией $pthread\_create()$ для создания потока и указания функции, которую начнет выполнять созданный поток. Далее при помощи вызова $pthread\_join()$ необходимо (в рамках данной лабораторной работы) дождаться завершения всех вспомогательных потоков, чтобы в главном потоке обработать результаты их работы.

\clearpage

\section{Сведения о модулях программы}

Данная программа разбита на следующие модули:

\begin{itemize}[label=---]
	\item \texttt{main.cpp} -- файл, содержащий точку входа в программу. В нем происходит общение с пользователем и вызов алгоритмов;
	\item \texttt{tf\_alg.cpp} –- файл содержит функции динамического выделения, очищения памяти и однопоточную реализацию алгоритма;
	\item \texttt{pthread\_tf.cpp} -- файл содержит функции многопоточной реализации алгоритма;
	\item \texttt{time\_meassure.cpp} –- файл содержит функции, замеряющее процессорное время методов однопоточной и многопоточной реализаций алгоритма.
\end{itemize}

\section{Реализация алгоритмов}

В листингах \ref{lst:tf_alg} -- \ref{lst:multy_tf_alg} приведены реализации алгоритма выделения наиболее информативных терминов для каждого документа.

В качестве терминов в данной реализации рассматриваются слова, состоящие из латинских букв. В качестве документов рассматриваются строки, состоящие из таких слов.

\clearpage

\lstinputlisting[label=lst:tf_alg,caption=Функция работы одного вспомогательного потока, firstline=85,lastline=121]{../src/tf_alg.c}

\clearpage

\lstinputlisting[label=lst:thread,caption=Функция работы одного вспомогательного потока, firstline=3,lastline=31]{../src/pthread_tf.c}

\clearpage

\lstinputlisting[label=lst:multy_tf_alg,caption=Функция работы основного потока запускающего вспомогательные потоки, firstline=33,lastline=59]{../src/pthread_tf.c}


\section*{Вывод}

В данном разделе были рассмотрены средства реализации, а также представлен листинг реализации алгоритма выделения наиболее информативных терминов для каждого документов.