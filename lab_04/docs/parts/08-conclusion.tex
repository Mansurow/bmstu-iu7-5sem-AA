\chapter*{Заключение}
\addcontentsline{toc}{chapter}{Заключение}

В ходе выполнения лабораторной работы было выявлено, что в результате использования многопоточной реализации время выполнения процессам может как улучшиться, так и ухудшиться в зависимости от количества используемых потоков.

Цель, поставленная в начале работы, была достигнута. 
Кроме того были достигнуты все поставленные задачи.
\begin{enumerate}
	\item Описаны основы распараллеливания вычислений.
	\item Разработано программное обеспечение, которое реализует однопоточный алгоритм вычисления термина частоты для классификации.
	\item Разработана и реализована многопоточная версия данного алгоритма.
	\item Определены средства программной реализации.
	\item Выполнены замеры процессорного времени работы реализаций алгоритма.
	\item Проведен сравнительный анализ по времени работы реализаций алгоритмов.
\end{enumerate}

При значении потоков превышающем число логических ядер (более 8 для устройства, на котором проводилось тестирование), затраты на содержание потоков превышают преимущество от использования многопоточности и время выполнения по сравнению с лучшим результатом (для 8 потоков) растут.