\chapter{Аналитическая часть}
В данном разделе была представлена информация о многопоточности и исследуемом алгоритме выделения терминов для классификации.

\section{Многопоточность}

\textbf{Многопоточность}~\cite{multithreading} \text{(англ. \textit{multithreading})} --- это способность центрального процессора (ЦП) обеспечивать одновременное выполнение нескольких потоков в рамках использования ресурсов одного процессора. Поток -- последовательность инструкций, которые могут исполняться параллельно с другими потоками одного и того же породившего их процесса.

Процессом~\cite{process} называют программу в ходе своего выполнения. Таким образом, когда запускается программа или приложение, создается процесс. Один процесс может состоять из одного или больше потоков.
Таким образом, поток является сегментом процесса, сущностью, которая выполняет задачи, стоящие перед исполняемым приложением. 
Процесс завершается тогда, когда все его потоки заканчивают работу.
Каждый поток в операционной системе является задачей, которую должен выполнить процессор. Сейчас большинство процессоров умеют выполнять несколько задач на одном ядре, создавая дополнительные, виртуальные ядра, или же имеют несколько физических ядер. Такие процессоры называются многоядерными. 

При написании программы, использующей несколько потоков, следует учесть, что при последовательных запуске потоков и передаче управления исполняемому потоку не получится раскрыть весь потенциал многопоточности, поскольку выигрыш от распараллеливания решение задач.
Необходимо создавать потоки для независимых по данным и выполнять их параллельно, тем самым сокращая общее время выполнения процесса.

Одной из проблем, встающих при использовании потоков, является проблема совместного доступа к информации. Фундаментальным ограничением является запрет на запись из двух и более потоков в одну ячейку памяти одновременно.

Из этого следует, что необходим примитив синхронизации обращения к данным -- так называемый мьютекс \text{(англ. \textit{mutex --- mutual exclusion})}. 
Он может быть захвачен для работы в монопольном режиме или освобожден мьютекса. 
Так, если 2 потока одновременно пытаются захватить мьютекс, успевает только один, а другой будет ждать освобождения. 

Набор инструкций, выполняемых между захватом и освобождением мьютекса, называется \textit{критической секцией}. 
Поскольку в то время, пока мьютекс захвачен, остальные потоки, требующие выполнения критической секции для доступа к одним и тем же данным, ждут освобождения мьютекса для его последующего захвата, требуется разрабатывать программное обеспечение таким образом, чтобы критическая секция была минимальной по объему.

\section{Термовая частота}

Классификация текстов является одной из основных задач компьютерной лингвистики, поскольку к ней сводится ряд других задач: определение тематической принадлежности текстов, автора текста, эмоциональной окраски высказываний и др. Для обеспечения информационной и общественной безопасности большое значение имеет анализ в телекоммуникационных сетях контента, содержащего противоправную информацию.~\cite{avto_lingv}

Под термом документов будем понимать все одиночные слова, встреченные в тексте хотя бы одного документа коллекции, за исключением стоп-слов, то есть распространенных слов, не характеризующих документы по смыслу, например, предлогов, союзов и т. п. Вдобавок, каждой встреченной форме слова, например, в разных падежах и числах, будет соответствовать один и тот же терм, например, данное слово в начальной форме.

Термовая частота (англ. TF)~\cite{lingv} --- это отношение числа вхождений $k$-го терма в документ к общему количеству термов в документе:
\begin{equation}
	TF_k = \frac{t_k}{n},
\end{equation}
где $t_k$ --- количество вхождений $k$-го терма в документ и $n$ --- общее количество термов документа.

В данной лабораторной работе проводится распараллеливания алгоритма выделения термов из выборки текстов на основе TF. Для этого все документы поровну распределяются между всеми потоками.

В качестве одного из аргументов каждый поток получает выделенный для него строку массива счетчиков TF длины $N_j$, где $N_j$ — это количество термов в документе. Так как каждая строка массива передается в монопольное использование каждому
потоку, не возникает конфликтов доступа к разделяемым ячейки памяти,следовательно, в использовании средства синхронизации в виде мьютекса нет необходимости.

\section*{Вывод}
В данном разделе была представлена информация о многопоточности и исследуемом алгоритме.