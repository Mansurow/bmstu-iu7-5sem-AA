\chapter*{Введение}
\addcontentsline{toc}{chapter}{Введение}

По мере развития вычислительных систем программисты столкнулись с
необходимостью производить параллельную обработку данных для улучшения отзывчивости системы, ускорения производимых вычислений и рационального использования вычислительных мощностей. Благодаря развитию
процессоров стало возможным использовать один процессор для выполнения нескольких параллельных операций, что дало начало термину «многопоточность».


Целью данной лабораторной работы является изучение принципов
и получение навыков организации параллельного выполнения операций.

Для поставленной цели необходимо выполнить следующие задачи.
\begin{enumerate}
	\item Изучение основы распараллеливания вычислений.
	\item Разработать программное обеспечение, которое реализует однопоточный алгоритм вычисления термина частоты для классификации.
	\item Разработать и реализовать многопоточную версию данного алгоритма.
	\item Определить средства программной реализации.
	\item Выполнить замеры процессорного времени работы реализаций алгоритма.
	\item Провести сравнительный анализ по времени работы реализаций алгоритмов.
\end{enumerate}