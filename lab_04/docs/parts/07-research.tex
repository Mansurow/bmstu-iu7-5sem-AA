\chapter{Исследовательская часть}

В данном разделе приведены технические характеристики устройства, на котором проводилось измерение времени работы программного обеспечения, а также результаты замеров времени.

\section{Технические характеристики}

Технические характеристики устройства, на котором выполнялись замеры по времени представлены далее.

\begin{itemize}[label=---]
	\item Процессор: Intel(R) Core(TM) i5-10300H CPU 2.50 ГГц~\cite{intel}.
	\item Количество ядер: 4 физических и 8 логических ядер.
	\item Оперативная память: 16 ГБайт.
	\item Операционная система: Windows 11 Pro 64-разрядная система~\cite{windows}.
\end{itemize}

При замерах времени ноутбук был включен в сеть электропитания и был нагружен только системными приложениями.

\section{Временные характеристики}

Для замеров времени использовалась функция получения значения системных часов $clock\_gettime()$~\cite{c-lang-time}. Функция применялась два раза --- в начале и в конце измерения времени, значения полученных временных меток вычитались друг из друга для получения времени выполнения программы.
Документы заполнялись случайными буквами латинского алфавита.
Замеры проводились по 1000 раз для набора значений количества потоков 0, 1, 2, 4, 6, 8, 16, 32, 64, где значение количества потоков 0 соответствует однопоточной программе, а значение 1 — программе, создающей один дополнительный поток, выполняющий все вычисления. 

В таблице  \ref{tbl:time_tf} представлены замеры времени выполнения программы в зависимости от количества потоков. Замеры проводились на документах, представленными строками длиной в 64 символа и диапазон количеств документов начиная с 512 до 6656 с шагом 512.

\begin{table}[ht]
	\small
	\begin{center}
		\begin{threeparttable}
		\caption{Результаты нагрузочного тестирования (в мкс)}
		\label{tbl:time_tf}
		\begin{tabular}{|c|c|c|c|c|c|c|c|c|c|}
			\hline
			\multirow{2}{*}{\bfseries Ко-во документов} & \multicolumn{9}{c|}{\bfseries Время, мкс} \\ \cline{2-10}
			 & \bfseries 0 & \bfseries 1 & \bfseries 2 & \bfseries 4 & \bfseries 6 & \bfseries 8 & \bfseries 16 & \bfseries 32 & \bfseries 64
			\csvreader{csv/times.csv}{}
			{\\\hline \csvcoli & \csvcolii & \csvcoliii & \csvcoliv & \csvcolv & \csvcolvi & \csvcolvii & \csvcolviii & \csvcolix & \csvcolx} \\
			\hline
		\end{tabular}
		\end{threeparttable}
	\end{center}
\end{table}

\begin{figure}[h!]
	\centering
	\begin{tikzpicture}
		\begin{loglogaxis}[	
			height = 0.4\paperheight, 
			width = 0.8\paperwidth,
			legend pos = north west,
			table/col sep=comma,
			xlabel={кол-во документов (ед.)},
			ylabel={время, мкс},
			]
			\legend{ 
				Без потоков,
				1 поток,
				2 потока,
				6 потоков,
				8 потоков,
				16 потоков,
				64 потока
			};
			\addplot [
			solid,
			thick, 
			draw = blue,
			mark = --, 
			mark options = {
				scale = 2, 
				fill = blue, 
				draw = black
			}
			] table [x=docs, y=0] {csv/times.csv};
			\addplot [
			dashed,
			thick,
			draw = green,
			mark = *, 
			mark options = {
				scale = 1, 
				fill = green, 
				draw = black
			}
			] table [x=docs, y=1] {csv/times.csv};
			\addplot [
			dash dot,
			thick, 
			draw = red,
			mark = *, 
			mark options = {
				scale = 1, 
				fill = red, 
				draw = black
			}
			] table [x=docs, y=2] {csv/times.csv};
			\addplot [
			dashed,
			thick,
			draw = orange,
			mark = --, 
			mark options = {
				scale = 1, 
				fill = orange, 
				draw = black
			}
			] table [x=docs, y=6] {csv/times.csv};
			\addplot [
			solid,
			thick,
			draw = purple,
			mark = cube*, 
			mark options = {
				scale = 1, 
				fill = purple, 
				draw = black
			}
			] table [x=docs, y=8] {csv/times.csv};
			\addplot [
			dash dot,
			thick, 
			draw = gray,
			mark = --, 
			mark options = {
				scale = 1, 
				fill = gray, 
				draw = black
			}
			] table [x=docs, y=16] {csv/times.csv};
			\addplot [
			solid, 
			thick,
			draw = black,
			mark = *, 
			mark options = {
				scale = 1, 
				fill = black, 
				draw = black
			}
			] table [x=docs, y=64] {csv/times.csv};						
		\end{loglogaxis}
	\end{tikzpicture}
	\caption{Результаты замеров времени работы реализаций алгоритма с разным количеством потоков в зависимости от количества документов}
	\label{img:g1}
\end{figure}

\clearpage

\begin{figure}[h!]
	\centering
	\begin{tikzpicture}
		\begin{axis}[	
			height = 0.4\paperheight, 
			width = 0.8\paperwidth,
			legend pos = north west,
			table/col sep=comma,
			xlabel={кол-во потоков (ед.)},
			ylabel={время, мкс},
			xmin = 1,
			xmax = 64
			]
			\addplot [
			solid,
			thick, 
			draw = blue,
			mark = --, 
			mark options = {
				scale = 2, 
				fill = blue, 
				draw = black
			}
			] table [x=thr, y=time] {csv/threads.csv};			
		\end{axis}
	\end{tikzpicture}
	\caption{Результаты замеров времени работы реализации алгоритма для 5632 документов в зависимости от количества потоков}
	\label{img:g2}
\end{figure}

Из полученных результатов можно сделать вывод, что однопоточный процесс работает быстрее процесса, создающего вспомогательный поток для обработки всех документов. Это связано с дополнительными временными затратами на создание потока и передачи ему необходимых аргументов.
Наилучший результат по времени для всех значений количества документов показал процесс с 8 дополнительными потоками, выполняющими вычисления. Рекомендуется использовать на данной архитектуре ЭВМ число дополнительных потоков равное числу логических ядер устройства.
Для числа потоков, большего данной величины, затраты на содержание потоков превышают преимущество от использования многопоточности, и функция времени от количества потоков начинает расти

\section{Вывод}
В результате экспериментов было выявлено, что использование многопоточности может уменьшить время выполнения реализации алгоритма по сравнению с однопоточной программой при условии использования подходящего количества потоков.

Выборка из результатов замеров времени (для 5632 документов):
\begin{itemize}
	\item однопоточный процесс --- 5580 мкс;
	\item один дополнительный поток, выполняющий все вычисления --- 5683 мкс;
	\item 8 потоков (лучший результат) — 3607 мкс, что в 1,57 раз быстрее выполнения однопоточного процесса;
	\item 64 потока (худший результат) — 6205 мкс, что в 0,89 раз медленнее выполнения однопоточного процесс.
\end{itemize}

Таким образом, использование дополнительных потоков может как ускорить выполнение процесса по сравнению с однопоточным процессом (в 1,57 раз для 8 потоков), так и увеличить время выполнения (в 0,89 раз для 64 потоков).
Рекомендуется использовать на данной архитектуре ЭВМ число дополнительных потоков, равное числу логических ядер устройства.