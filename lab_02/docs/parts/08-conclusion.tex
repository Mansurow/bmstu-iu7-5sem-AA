\chapter*{Заключение}
\addcontentsline{toc}{chapter}{Заключение}

В данной лабораторной работе были рассмотрены алгоритмы умножения матриц. В программировании, как и в математике, часто приходится прибегать к использованию матриц. Существует огромное количество областей их применения в этих сферах. Матрицы активно используются при выводе различных формул в физике, таких, как:
\begin{itemize}
	\item градиент;
	\item дивергенция;
	\item ротор.
\end{itemize}

Нельзя обойти стороной и различные операции над матрицами – сложение, возведение в степень, умножение. При различных задачах размеры матрицы могут достигать больших значений, поэтому оптимизация операций работы над матрицами является важной задачей в программировании. В данной лабораторной работе пойдёт речь об оптимизациях операции умножения матриц.

Цель, которая была поставлена в начале лабораторной работы была достигнута, а также в ходе выполнения лабораторной работы были решены следующие задачи:
\begin{enumerate}[label={\arabic*)}]
	\item изучены алгоритмы умножения матриц;
	\item создано ПО, реализующее следующие алгоритмы:
	\begin{itemize}
		\item классический алгоритм умножения матриц;
		\item алгоритм Винограда;
		\item оптимизированный алгоритм Винограда.
	\end{itemize}
	\item Оценено трудоемкости сортировок;
	\item Проведено анализ затрат работы программы по времени, выяснить влияющие на них характеристики.
	\item Проведено сравнительный анализ между алгоритмами.
\end{enumerate}