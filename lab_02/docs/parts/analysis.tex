\chapter{Аналитическая часть}

В этом разделе будут рассмотрены классический алгоритм умножения матриц и алгоритм Винограда, а также его оптимизированная версия.

\section{Матрица}

\textbf{Матрицой} [\ref{lib:matrix}] называют таблицу чисел $a_{ik}$ вида
\begin{equation}
	\begin{pmatrix}
		a_{11} & a_{12} & \ldots & a_{1n}\\
		a_{21} & a_{22} & \ldots & a_{2n}\\
		\vdots & \vdots & \ddots & \vdots\\
		a_{m1} & a_{m2} & \ldots & a_{mn}
	\end{pmatrix},
\end{equation}

состоящую из $m$ строк и $n$ столбцов. Числа $a_{ik}$ называются её \textit{элементами}.

Пусть $A$ -- матрица, тогда $A_{i,j}$ -- элемент этой матрицы, который находится на \textit{i-ой} строке и \textit{j-ом} столбце.

Можно выделить следующие операции над матрицами:
\begin{enumerate}[label=\arabic*)]
	\item сложение матриц одинакового размера;
	\item вычитание матриц одинакового размера;
	\item умножение матриц в случае, если количество столбцов первой матрицы равно количеству строк второй матрицы. В итоговой матрице количество строк будет, как у первой матрицы, а столбцов -- как у второй. \newline
\end{enumerate}

\textit{Замечание:} операция умножения матриц не коммутативна -- если \textit{A} и \textit{B} -- квадратные матрицы, а \textit{C} -- результат их перемножения, то произведение \textit{AB} и \textit{BA} дадут разный результат \textit{C}.

\section{Классический алгоритм умножения матриц}

Пусть даны две матрицы

\begin{equation}
	A_{lm} = \begin{pmatrix}
		a_{11} & a_{12} & \ldots & a_{1m}\\
		a_{21} & a_{22} & \ldots & a_{2m}\\
		\vdots & \vdots & \ddots & \vdots\\
		a_{l1} & a_{l2} & \ldots & a_{lm}
	\end{pmatrix},
	\quad
	B_{mn} = \begin{pmatrix}
		b_{11} & b_{12} & \ldots & b_{1n}\\
		b_{21} & b_{22} & \ldots & b_{2n}\\
		\vdots & \vdots & \ddots & \vdots\\
		b_{m1} & b_{m2} & \ldots & b_{mn}
	\end{pmatrix},
\end{equation}

тогда матрица $C$
\begin{equation}
	C_{ln} = \begin{pmatrix}
		c_{11} & c_{12} & \ldots & c_{1n}\\
		c_{21} & c_{22} & \ldots & c_{2n}\\
		\vdots & \vdots & \ddots & \vdots\\
		c_{l1} & c_{l2} & \ldots & c_{ln}
	\end{pmatrix},
\end{equation}

где
\begin{equation}
	\label{eq:M}
	c_{ij} =
	\sum_{r=1}^{m} a_{ir}b_{rj} \quad (i=\overline{1,l}; j=\overline{1,n})
\end{equation}

будет называться произведением матриц $A$ и $B$ [\ref{lib:matrix}].

Классический алгоритм реализует данную формулу.

\section{Алгоритм Копперсмита-Винограда}



\section{Алгоритм Копперсмита-Винограда}

\section{Оптимизация алгоритма Копперсмита-Винограда}