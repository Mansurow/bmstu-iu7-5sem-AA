\chapter*{Введение}
\addcontentsline{toc}{chapter}{Введение}

Оптимизации, позволяющие улучшить работу существующих алгоритмов
или помогающие решить поставленную задачу иным, более эффективным
способом, были важны во все времена. Одной из важных задач являются
задачи поисков оптимальных маршрутов.

Целью данной лабораторной работы является параметризация метода решения задачи коммивояжера на основе муравьиного метода.

Для поставленной цели необходимо выполнить следующие задачи.
\begin{enumerate}
	\item Описать задачу коммивояжера.
	\item Описать методы решения задачи коммивояжера --- метод полного перебора и метод на основе муравьиного алгоритма.
	\item Привести схемы муравьиного алгоритма и алгоритма, позволяющего решить задачу коммивояжера методом полного перебора.
	\item Разработать и реализовать программный продукт, позволяющий решить задачу коммивояжера исследуемыми методами.
	\item Сравнить по времени метод полного перебора и метод на основе муравьиного алгоритма.
	\item Описать и обосновать полученные результаты в отчете о выполненной лабораторной работе.
\end{enumerate}

Выданный индивидуальный вариант для выполнения лабораторной работы:
\begin{itemize}
	\item неориентированы граф;
	\item без элитных муравьев;
	\item гамильтонов цикл;
	\item 80 дней вокруг света.
\end{itemize}