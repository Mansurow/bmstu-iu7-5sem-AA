\chapter{Технологическая часть}

В данном разделе рассмотрены средства реализации, а также представлены листинги реализаций алгоритма расчета термовой частота для всех термов из выборки документов.

\section{Средства реализации}

В данной работе для реализации был выбран язык программирования $Python$ \cite{python-lang}. В текущей лабораторной работе требуется замерить процессорное время работы выполняемой программы. 
Инструменты для этого присутствуют в выбранном языке программирования.

Время работы было замерено с помощью функции \textit{process\_time(...)} из библиотеки $time$ \cite{python-lang-time}.


\section{Сведения о модулях программы}

Данная программа разбита на следующие модули:
\begin{itemize}
	\item $main.py$ --- файл, содержащий точку входа;
	\item $menu.py$ --- файл, содержащий код меню программы;
	\item $utils.py$ --- файл, содержащий служебные алгоритмы;
	\item $constants.py$ --- файл, содержащий константы программы;
	\item $algorythms.py$ --- файл, содержащий код всех алгоритмов. 
\end{itemize}

\section{Реализация алгоритмов}

В листинге \ref{lst:full_comb} представлен реализация алгоритм полного перебора путей, а в листингах \ref{lst:ant}--\ref{lst:update} --- муравьиный алгоритм и дополнительные к нему функции.


\lstinputlisting[label=lst:full_comb,caption=Реализаиция алгоритма полного перебора, firstline=6,lastline=28]{../src/algorithms.py}

\clearpage

\lstinputlisting[label=lst:ant,caption=Реализаиция муравьиного алгоритма, firstline=120,lastline=146]{../src/algorithms.py}

\clearpage

\lstinputlisting[label=lst:find-way,caption=Реализация алгоритма нахождения массива вероятностей переходов в непосещенные города, firstline=85,lastline=101]{../src/algorithms.py}

\lstinputlisting[label=lst:calc-phero,caption=Реализация алгоритма нахождения массива вероятностей переходов в непосещенные города, firstline=40,lastline=43]{../src/algorithms.py}

\lstinputlisting[label=lst:choose-next,caption=Реализация алгоритма выбора следующего города, firstline=107,lastline=116]{../src/algorithms.py}

\lstinputlisting[label=lst:update,caption=Реализация алгоритма обновления матрицы феромонов, firstline=67,lastline=82]{../src/algorithms.py}

\clearpage

\section{Функциональные тесты}

В таблице \ref{tbl:functional_test} приведены тесты для функций программы. Все функциональные тесты пройдены \textit{успешно}.

\begin{center}
	\captionsetup{justification=raggedright,singlelinecheck=off}
	\begin{longtable}[c]{|c|c|c|c|c|}
		\caption{Функциональные тесты\label{tbl:functional_test}} \\ \hline
		Матрица смежности & Ожидаемый результат & Результат программы \\
		\hline
		$ \begin{pmatrix}
			0 &  4 &  2 &  1 & 7 \\
			4 &  0 &  3 &  7 & 2 \\
			2 &  3 &  0 & 10 & 3 \\
			1 &  7 & 10 &  0 & 9 \\
			7 &  2 &  3 &  9 & 0
		\end{pmatrix}$ &
		15, [0, 2, 4, 1, 3, 0] &
		15, [0, 2, 4, 1, 3, 0] \\
		
		$ \begin{pmatrix}
			0 & 1 & 2 \\
			1 & 0 & 1 \\
			2 & 1 & 0	
		\end{pmatrix}$ &
		4, [0, 1, 2, 0] &
		4, [0, 1, 2, 0] \\
		
		$ \begin{pmatrix}
			0 & 15 & 19 & 20 \\
			15 &  0 & 12 & 13 \\
			19 & 12 &  0 & 17 \\
			20 & 13 & 17 &  0
		\end{pmatrix}$ &
		64, [0, 1, 2, 3, 0] &
		64, [0, 1, 2, 3, 0] \\
		\hline
	\end{longtable}
\end{center}

\section*{Вывод}

Были представлены листинги всех реализаций алгоритмов --- полного перебора и муравьиного. Также в данном разделе была приведена информации о выбранных средствах для разработки алгоритмов и сведения о модулях программы, проведено функциональное тестирование.
