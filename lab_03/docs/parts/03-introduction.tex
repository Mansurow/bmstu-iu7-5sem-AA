\chapter*{Введение}
\addcontentsline{toc}{chapter}{Введение}

В данной лабораторной работе будет рассмотрены сортировки. 

Сортировка - перегруппировка некой последовательности, или кортежа, в определенном порядке. Это одна из главных процедур обработки структурированных данных. Расположение элементов в определенном порядке позволяет более эффективно проводить работу с последовательностью данных, в частности при поиске некоторых данных.

Различают два вида сортировок: сортировку массивов и сортировку файлов. Сортировку массивов также называют внутренней, т.к. все элементы массивов хранятся в быстрой внутренней памяти машины с прямым доступом, а сортировку файлов – внешней, т.к. их элементы хранятся в медленной, но более емкой внешней памяти. При внутренней сортировке доступ к элементам может осуществляться в
произвольном порядке. Напротив, при внешней сортировке доступ к элементам производится в строго определенной последовательности

Существует множество алгоритмов сортировки, но любой алгоритм сортировки имеет:
\begin{itemize}
	\item сравнение, которое определяет, как упорядочена пара элементов;
	\item перестановка для смены элементов местами;
	\item алгоритм сортировки, использующий сравнение и перестановки.
\end{itemize}

Каждый алгоритм имеет свои достоинства, но в целом его оценка зависит от ответов, которые будут получены на следующие вопросы:
\begin{itemize}
	\item с какой средней скоростью этот алгоритм сортирует информацию;
	\item какова скорость для лучшего и для худшего случаев;
	\item является ли естественным “поведение” алгоритма (т.е. возрастает ли скорость сортировки с увеличением упорядоченности массива);
	\item является ли алгоритм стабильным (т.е. выполняется ли перестановка элементов с одинаковыми значениями).
\end{itemize}

Целью данной лабораторной работы является описание и исследование трудоемкости алгоритмов сортировки.

Для поставленной цели необходимо выполнить следующие задачи.
\begin{enumerate}
	\item Описать расстояния Левенштейна и Дамерау-Левенштейна.
	\item Создать программное обеспечение, реализующее следующие алгоритмы сортировки:
	\begin{itemize}
		\item Шелла;
		\item Пирамидальная;
		\item Бусинами.
	\end{itemize}
	\item Оценить трудоемкости сортировок.
	\item Замерить время реализации.
	\item Провести анализ затрат работы программы по времени, выяснить влияющие на них характеристики.
\end{enumerate}