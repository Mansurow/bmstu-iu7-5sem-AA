\chapter{Технологическая часть}

В данном разделе приведены требования к программному обеспечению, средства реализации, листинг кода и функциональные тесты.

\section{Требования к программному обеспечению}


К программе предъявлены ряд требований:

\begin{itemize}[label=---]
	\item должна иметь интерфейс для выбора действий;
	\item должна динамически выделять память под массив данных;
	\item должна замерять процессорное время алгоритмов сортировки.
\end{itemize}

\section{Средства реализации}

В качестве языка программирования для реализации данной лабораторной работы был выбран язык $C++$~\cite{cpp-lang}. Данный выбор обусловлен наличием у языка
встроенной библиотекой измерения процессорного времени и соответствием с выдвинутыми техническими требованиям.

В качестве измерения время использовалось функция \textit{clock()} из библиотеки \textit{ctime}~\cite{cpp-lang-time}. 

\clearpage

\section{Сведения о модулях программы}

Данная программа разбита на следующие модули:

\begin{itemize}[label=---]
	\item \texttt{main.cpp} -- файл, содержащий точку входа в программу. В нем происходи	общение с пользователем и вызов алгоритмов;
	\item \texttt{sorts.cpp} –- файл содержит алгоритмы сортировка.
	\item \texttt{allocate.cpp} –- файл содержит функции динамического выделения и очищения памяти для матрицы.
	\item \texttt{generate\_array.cpp} -- файл содержит функции генерации данных для массива.
	\item \texttt{read\_array.cpp} -- файл содержит функции чтение массива потока данных.
	\item \texttt{print\_array.cpp} -- файл содержит функцию вывода массива данных.
	\item \texttt{cpu\_time.cpp} –- файл содержит функции, замеряющее процессорное время методов сортировки.
\end{itemize}

В листингах \ref{lst:shell} -- \ref{lst:bead} реализованы алгоритмы сортировка, а именно сортировка Шелла, пирамидальная сортировка и сортировка бусинами.
В листингах \ref{lst:allocate} -- \ref{lst:swap} реализованы алгоритмы выделение, вывода матриц и алгоритм перестановки элементов местами.

\clearpage

\begin{lstlisting}[label=lst:shell,caption=Функция сортировки методом Шелла]
void shell_sorting(int *array, size_t n)
{
	if (array != nullptr && n > 0)
		for (int s = n / 2; s > 0; s /= 2)
			for (int i = s, j; i < n; i++)
			{
				int temp = array[i];
				for (j = i; array[j - s] > temp && j >= s; j -= s)
				array[j] = array[j - s];
				array[j] = temp;
			}
}
\end{lstlisting}

\clearpage

\begin{lstlisting}[label=lst:heap,caption=Функция пирамидальной сортировки]
void heapify(int *array, size_t i, size_t n) {
	size_t largest;
	int done = 0;
	size_t l, r;
	
	while((2 * i <= n) && !done) {
		l = 2 * i;
		r = 2 * i + 1;
		
		if (l == n) {
			largest = l;
		} else if (array[l] > array[r])
			largest = l;
		else
			largest = r;
		
		if (array[i] < array[largest]) {
			swap(array[i], array[largest]);
			i = largest;
		}
		else
			done = 1;
	}
}

void heap_sort(int *array, size_t n) {
	if (array != nullptr && n > 0) {
		for (int i = n / 2; i >= 0; i--) {
			heapify(array, i, n - 1);
		}
		
		for (int i = n - 1; i >= 1; i--) {
			swap(array[0], array[i]);
			heapify(array, 0, i - 1);
		}
	}
}
\end{lstlisting}

\clearpage

\begin{lstlisting}[label=lst:bead,caption=Функция сортировки бусинами]
void bead_sort(int *array, size_t n) {
	if (array != nullptr && n > 0) {
		int i, j, max, sum;
		unsigned char *beads;
		
		for (i = 1, max = array[0]; i < n; i++)
			if (array[i] > max)
				max = array[i];
		
		beads = static_cast<unsigned char *>(calloc(1, max * n));
		
		for (i = 0; i < n; i++)
			for (j = 0; j < array[i]; j++)
			beads[i * max + j] = 1;
		
		for (j = 0; j < max; j++) {
			for (sum = i = 0; i < n; i++) {
				sum += beads[i * max + j];
				beads[i * max + j] = 0;
			}
			for (i = n - sum; i < n; i++)
				beads[i * max + j] = 1;
		}
		
		for (i = 0; i < n; i++) {
			for (j = 0; j < max && beads[i * max + j]; j++);
			array[i] = j;
		}
		free(beads);
	}
}
\end{lstlisting}

\clearpage

\begin{lstlisting}[label=lst:allocate,caption=Функции выделение и освобождение памяти под массив]
int *allocate_arr(size_t n)
{
	int *arr = nullptr;
	if (n > 0)
		arr = static_cast<int *>(calloc(n, sizeof(int)));
	
	return arr;
}

void free_arr(int *arr, size_t n)
{
	if (n > 0)
	free(arr);
}
\end{lstlisting}

\begin{lstlisting}[label=lst:print,caption=Функция вывода print\_arr]
void print_array(int *array, size_t n)
{
	for(int i = 0; i < n; i++)
		std::cout << array[i] << " ";
	std::cout << std::endl;
}
\end{lstlisting}

\begin{lstlisting}[label=lst:swap,caption=Функция перестановки элементов места swap]
void swap(int &el1, int &el2)
{
	int temp = el1;
	el1 = el2;
	el2 = temp;
}
\end{lstlisting}

\clearpage

\section{Функциональные тесты}

В таблице \ref{tbl:func_tests} приведены функциональные тесты для разработанных алгоритмов сортировки. Все тесты пройдены успешно.
Под обозначением "$-$" \newline в строках таблицы значит, что данная алгоритм сортировки не может работать с отрицательными числами, а именно сортировка Бусинами работающая только с натуральными числами.

\begin{table}[ht]
	\small
	\begin{center}
		\begin{threeparttable}
		\caption{Функциональные тесты}
		\label{tbl:func_tests}
		\begin{tabular}{|c|c|c|c|c|}
			\hline
			\bfseries Массив
			& \bfseries Размер
			& \bfseries Ожидаемый р-т
			& \multicolumn{2}{c|}{\bfseries Фактический результат} \\ \cline{4-5}
			& & & \bfseries Шелла/Пирам. & \bfseries Бусинами \\
			\hline
			41 67 34 0 69 & 5 & 0 34 41 67 69 & 0 34 41 67 69 & 0 34 41 67 69 \\
			\hline
			31 57 24 -10 59 & 5 & -10 24 31 57 59 & -10 24 31 57 59 & - \\
			\hline
			1 2 3 4 5 & 5 & 1 2 3 4 5 & 1 2 3 4 5 & 1 2 3 4 5 \\
			\hline
			100 88 76 65 43 & 5 & 43 65 76 88 100 & 43 65 76 88 100 & 43 65 76 88 100 \\
			\hline
			-59 -33 -66 -100 -31 & 5 & -100 -66 -59 -33 -31 & -100 -66 -59 -33 -31 & - \\
			\hline
		\end{tabular}	
		\end{threeparttable}	
	\end{center}
\end{table}

\section*{Вывод}

Были реализованы алгоритмы: сортировка Шелла, пирамидальная сортировка и сортировка бусинами. Проведено тестирование разработанных алгоритмов.