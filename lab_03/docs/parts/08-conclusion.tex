\chapter*{Заключение}
\addcontentsline{toc}{chapter}{Заключение}

Исходя из полученных результатов замеров по времени, сортировка бусинами в любом случае работает дольше, чем сортировка Шелла и пирамидальная сортировка, а именно 100-200 раз хуже. Сравнение результатов замеров по времени сортировки Шелла и пирамидальной сортировки показали, что сортировка Шелла работает 2 раза быстрее, чем пирамидальная сортировка на отсортированных данных, на случайных данных сортировка Шелла работает в 1,5 раза лучше, чем пирамидальная сортировка, но на отсортированных данных по убывания пирамидальная сортировка работает в 1,5 раза лучше, чем сортировка Шелла.

Целью данной лабораторной работы является изучение и исследование трудоемкости алгоритмов сортировки.

Для поставленной цели были выполнены следующие задачи.
\begin{enumerate}[label={\arabic*)}]
	\item Описаны расстояния Левенштейна и Дамерау-Левенштейна;
	\item Создано программное обеспечение, реализующее следующие алгоритмы сортировки:
	\begin{itemize}[label=---]
		\item Шелла;
		\item Пирамидальная;
		\item Бусинами;
	\end{itemize}
	\item Оценены трудоемкости сортировок;
	\item Замерено время реализации;
	\item Проведен анализ затрат работы программы по времени, выяснить влияющие на них характеристики.
\end{enumerate}
