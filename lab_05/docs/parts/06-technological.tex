\chapter{Технологическая часть}

В данном разделе рассмотрены средства реализации, а также представлены листинги реализаций алгоритма расчета термовой частота для всех термов из выборки документов.

\section{Средства реализации}

Для данной лабораторной работы был выбран язык программирования $C++$~\cite{cpp-lang}. 
Данный выбор обусловлен наличием у языка встроенным модулем измерения процессорного времени~\cite{cpp-time} и предоставлением возможности:
\begin{itemize}
	\item работы с <<зеленными>> потоками предоставляется классом $thread$ \cite{thread};
	\item работы с мьютексами предоставляется классом $mutex$ \cite{mutex};
	\item работы с очередями предоставляется классом $queue$ \cite{queue}.
\end{itemize}.

\section{Сведения о модулях программы}

Данная программа разбита на следующие модули:

\begin{itemize}[label=---]
	\item \texttt{main.cpp} --- файл, содержащий точку входа в программу. В нем происходит общение с пользователем и вызов алгоритмов;
	\item \texttt{mtr\_op.cpp} --- файл содержит функции операций над матрицей и матрицей в РСФ;
	\item \texttt{read\_size.cpp} --- файл содержит функции чтения данных;
	\item \texttt{conveyor.cpp} --- файл содержит функции конвейерной обработки;
	\item \texttt{meassure.cpp} --- файл содержит функции, замеряющее процессорное время работы реализаций алгоритмов.
\end{itemize}

\section{Реализация алгоритмов}

На листинге \ref{lst:linear} представлена последовательная реализация конвейерной обработки.
На листингах \ref{lst:main_thr} -- \ref{lst:thr_3} представлены параллельная реализация конвейерной обработки и реализация каждой стадии в отдельном вспомогательном потоке.
На листингах \ref{lst:sum_1} -- \ref{lst:unpack} представлены реализации алгоритмов генерации матрицы РСФ, суммы двух матриц РСФ и распаковка результирующей матрицы РСФ в классическое представление матрицы.

\lstinputlisting[label=lst:linear,caption=Реализация последовательной конвейерной обработки, firstline=20,lastline=48]{../src/conveyor.cpp}

\clearpage

\lstinputlisting[label=lst:main_thr,caption={Реализация основного потока для  конвейерной обработки, создающий вспомогательные потоки}, firstline=99,lastline=121]{../src/conveyor.cpp}
\clearpage

\lstinputlisting[label=lst:thr_1,caption={Реализация вспомогательного потока, отвечающий за создание матриц РСФ}, firstline=50,lastline=62]{../src/conveyor.cpp}

\lstinputlisting[label=lst:thr_2,caption={Реализация вспомогательного потока, отвечающий за сложение матриц РСФ}, firstline=64,lastline=81]{../src/conveyor.cpp}

\clearpage

\lstinputlisting[label=lst:thr_3,caption={Реализация вспомогательного потока, отвечающий за распаковку матрицы (Часть 1)}, firstline=83,lastline=97]{../src/conveyor.cpp}

\lstinputlisting[label=lst:sum_1,caption={Реализация алгоритма генерации данных для матрицы РСФ}, firstline=57,lastline=75]{../src/mtr_op.cpp}

\lstinputlisting[label=lst:sum_2,caption={Реализация алгоритма генерации данных для матрицы РСФ (Часть 2)}, firstline=76,lastline=99]{../src/mtr_op.cpp}

\clearpage

\lstinputlisting[label=lst:gen,caption={Реализация алгоритма генерации данных для матрицы в РСФ}, firstline=3,lastline=33]{../src/mtr_op.cpp}

\clearpage

\lstinputlisting[label=lst:unpack,caption={Реализация алгоритма распоковки матрицы РСФ}, firstline=35,lastline=55]{../src/mtr_op.cpp}

\section*{Вывод}

В данном разделе была приведена информация о выбранных средствах для разработки алгоритмов. 
Для реализации алгоритмов был выбран язык С++.
Были представлены листинги для каждой из реализаций работы конвейера и его трех стадий, а именно генерации данных для двух матриц РСФ, сложение двух матриц РСФ и распаковка матрицы РСФ в классическое матричное представление.