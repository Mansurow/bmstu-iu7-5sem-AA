\chapter*{Введение}
\addcontentsline{toc}{chapter}{Введение}

Использование параллельной обработки открывает новые способы для ускорения работы программ.
Конвейерная обработка является одним из примеров, где использование принципов параллельности помогает ускорить обработку данных. 
Суть та же, что и при работе реальных конвейерных лент --- материал (данное) поступает на обработку, после окончания обработки материал передается на место следующего обработчика, при этом предыдущий обработчик не ждет полного цикла обработки материала, а получает новый материал и работает с ним.

Целью данной лабораторной работы является описание параллельных конвейерных вычислений.

Для поставленной цели необходимо выполнить следующие задачи.
\begin{enumerate}
	\item Описать организацию конвейерной обработки данных.
	\item Описать алгоритмы обработки данных, которые будут использоваться в текущей лабораторной работе.
	\item Реализовать программу, выполняющую конвейерную обработку с количеством лент не менее трех в однопоточной и многопоточной реализаций.
	\item Сравнить и проанализировать реализации алгоритмов по затраченному времени.
\end{enumerate}